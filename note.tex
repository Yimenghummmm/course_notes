\documentclass[11pt]{article}
\usepackage{geometry}
\geometry{
    a4paper,
    left=1.5cm,
    right=1.5cm,
    top=2cm
}
\usepackage{fancyhdr}
\setlength{\headheight}{15,2pt}
\pagestyle{fancy}
\fancyhead[LE,RO]{Yimeng Hu}
\fancyhead[RE,LO]{notes (math 323)}
\fancyfoot[CE,CO]{\thepage}
\usepackage{graphicx}
\graphicspath{{./image/}}
\usepackage{amsmath}
\usepackage{amsfonts}
\usepackage{amsthm}
\newtheorem{theorem}{\framebox{Thm}}[section]
\newtheorem{corollary}{\framebox{Cor}}[theorem]
\newtheorem{lemma}{\framebox{Lemma}}[theorem]
\newtheorem*{remark}{Remark}
\newtheorem*{note}{Note}
\newtheorem*{recall}{Recall}
\newtheorem*{solution}{\framebox{Sol}}
\newtheorem{definition}{\framebox{DEF}}[section]
\newtheorem{example}{\framebox{Ex}}[section]
\newcommand\independent{\protect\mathpalette{\protect\independenT}{\perp}}
\def\independenT#1#2{\mathrel{\rlap{$#1#2$}\mkern2mu{#1#2}}}
\renewcommand{\proofname}{$\underline{{\textit{PROOF:}}}$}
\newcommand\tab[1][1cm]{\hspace*{#1}}
\usepackage{hyperref}

\begin{document}
\author{Prof: David Wolfson \\Yimeng Hu }
\title{Statistics (Math 323) winter 2019}
\date{}
\maketitle
\tableofcontents
\newpage
    \section{Lec 01, Jan 08}
        \subsection{Introduction}
            Why study probability? 
            \begin{itemize}
                \item as a discipline in its own right.
                \item as a part of mathematics/ applied mathematics
                \item Most importantly, as a tool for statistical inference
            \end{itemize}
            \textbf{The meaning of probability} (i.e when we say "the probability" of an event A is 2/3, what do we mean)
            \framebox{Ex} a box has 6 Red and 4 green marbles. Draw a marble at random from the box.
                            What is the probability \tab that the marble is red?
                \begin{enumerate}
                    \item If we say P(red) = $\frac{6}{10}$ what do we mean by this statement?
                        \begin{solution}\tab \\
                            We cannot simply define "the probability" of getting a red as 
                            $\lim_{N \rightarrow \infty} \frac{\text{\# of red}}{\text{\# of trials}}$
                            since we do not know if this limit will exist and be unique for every sequence of trials.
                            So instead in the 1930. The Great Russian mathematician A.N. Komolgorov proposed \textbf{3 axioms/assumptions}
                            that probability should satisfies and then developped a theory of probability from these.
                        \end{solution}
                        \begin{note}
                            As a consequence of The law of large number, we interpret probability as a limiting relative frequency.
                            However it has nothing to do with relative frequencies.
                        \end{note}
                        
                    \item How did you arrive at this answer?
                        \begin{solution} \tab \\
                            In order to arrive at the answer $\frac{6}{10}$, we need to use the Komolgorov axioms
                            and any theorm that follows from that to prove that this is indeed correct.
                        \end{solution}
                \end{enumerate}
        \subsection{Basic Set Algebra}
            \begin{enumerate}
                \item $A = \{\omega : \omega \in A\}$ where A is an event (a set) consist of elementary outcomes w.
                \item $A \cap B = \{\omega: \omega \in A \text{and} \omega \in B\}$ \tab \textbf{note:} "and" $\Rightarrow$ intersection
                \item $A \cup B = \{\omega: \omega \in A \text{or} \omega \in B\}$ \tab \textbf{note:} "or" $\Rightarrow$ union.
                \item $A \subset B:= \omega \in A \Rightarrow \omega \in B$
                \item All discussion take place in the context of the \underline{\textbf{universal set}} \textit{S}
                \item $A^c := \{\omega \in S : \omega \notin A\}$ \tab \textbf{note:} It is sometimes easier to first find $P(A^c)$
                \item $A\cap B = \phi \Rightarrow$ A and B disjoint or mutually exclusive
                \item De Morgan's Law: 
                    \begin{itemize}
                        \item[a)]
                            $ (A\cap B)^c = A^c \cup B^c$
                        \item[b)] 
                            $ (A\cup B)^c = A^c \cap B^c$ 
                    \end{itemize}
            \end{enumerate}
    \newpage
    \section{Lec 02, Jan 10}
            \subsection{Experiment}
                \begin{definition}
                    An \textbf{\textit{experiment}} is defined informally as the performance of some actions
                \end{definition} 

                \begin{definition}
                    An \textbf{\textit{Random Experiment}} is one for which the outcome are not known in advance. 
                    i.e there is uncertainty in the outcome that will be observed. 
                    (Once thet experiment has been conducted though you may not know the outcome, there is nothing random about the outcome)
                \end{definition}

                \begin{example}
                         Toss a coin twice and observe the outcome
                         The pre-experiment outcomes are random / uncertain 
                \end{example}

                \begin{example}
                    Take 60 subjects who will undergo surgery for a certain disorder before we observe their 
                    $\underbrace{\text{time to recovery}}_{outcome}$, are random/uncertain
                \end{example}

                \begin{example}
                    Toss a coin until you observe the first head. Let the trial at which this happens be the outcome of interest.
                    This outcome is random before you start tossing.
                \end{example}
            
            \subsection{Sample Space}
                \begin{definition}
                    The set of all possible outcome of an experiment is called the \textbf{\textit{Sample Space (S)}}
                    of the experiment. We denote each outcome as $\omega$, an elemenatry outcome.
                \end{definition}

                \begin{note}
                    a sample space mainly depend on how you define your outcomes.
                \end{note}

                \begin{example}
                    Draw a marble at random from 6 Red and 4 Green. 
                    
                    \begin{enumerate}
                        \item If order \textbf{does not} matter,
                            Let $w_1$:= event which a red marble is drawn,
                            $w_2$ := event which a green marble is drawn. then the sample space is as following:
                            $S = \{w_1, w_2\}$
                        \item If order matters, number the marbles WLOG $\{\underbrace{1,...,6}_{Red},\underbrace{,7,...,10}_{Green}\}$
                            Let $w_i:=$ event which marble i is drawn for i = 1,...,10
                            Then the sample space is: S = \{$w_1,...,w_{10}$\}
                    \end{enumerate}
                \end{example}

                \begin{example}
                    Suppose there are n people in a room, ask these people when their birthday are.
                    Let S be the set of outcomes that we could get at the completion of our experiment. 
                    Then S could be defined as: 
                    \\$S = \{\underbrace{\{Jan 1, ..., Jan 1\}}_{w_1},..., \underbrace{\{Dec 31, ..., Dec 31\}}_{w_n}\}$
                \end{example}
                \begin{note}
                    Here the elementary outcomes are n-dim vectors
                \end{note}

                \begin{example}
                    Toss a coin until you observe the $1^{st}$ head. Let n be the trial number at which this occurs.
                    Then S = \{$w_1,...,w_n$\} = \{1,...,n\}. S is an example of sample space with a countably many numbers of possible outcomes.
                \end{example}

                \begin{example}
                    Suppose that you measure the height of a dam every July $1^st$. The set of possible heights might be 
                    S = \{[0,20]\} where 20 is the height of the dam wall in meters.
                    Here S is an uncountably infinite set. \textbf{\textit{Note: }}In real lift no such sample space exits.
                \end{example}
            \subsection{Kolmogorov Axioms}
                \begin{definition}
                    Any subset $E \subset S $ is defined as an \textbf{\textit{event}}.
                     The empty set $\phi \subset S$ is also an event.
                \end{definition}
                
                \begin{definition}
                    a function P() is a set function on the subset of S if P(A) is a real number for every subset of S.
                    Let S be a sample space. Then P() is called a probability measure if P is a real valued set function on S s.t
                    \begin{enumerate}
                        \item $\forall E \subset S, P(E) \geq 0$
                        \item P(S) = 1
                        \item Let $E_1, E_2, ...$ be any countable connection of events such that they are mutually exclusive, 
                        \\i.e $E_i \cap E_j = \phi \tab \forall i\neq j$. 
                         Then P($\bigcup\limits_{i=1}^{\infty}E_i$) = $\sum\limits_{i=1}^{\infty}P(E_i)$ 
                    \end{enumerate}
                \end{definition}
                \begin{note}
                    From these 3 axioms, we developpe the entire theory of probability including the Law of Large numbers 
                    which allows us to interpret probability as a limiting relative frequency. 
                    We shall state and prove 5 theorem that will be useful for solving word problems.
                \end{note}
                \newpage
    \section{Lec 03, Jan 15}
        \subsection{The 5 theorems}
            \begin{theorem}
                \begin{enumerate}
                    \item For any event A, P($A^c$) = 1-P(A)
                        \begin{proof}
                            \begin{align*}
                                 A \cup A^c = S,&\Rightarrow P(A\cup A^c) = P(S) = 1 \tag{Ax 2} \\
                                 A \cap A^c = \phi &\Rightarrow P(A \cup A^c) = P(A) + P(A^c) \tag{Ax 3}\\
                                 \Rightarrow P(A^c) &= 1 - P(A)
                            \end{align*}
                        \end{proof}
                    \item P($\phi$) = 0 
                    \item P($A \cap B^c) = P(A) - P(A \cap B )$
                        \begin{proof}
                            trick: Try to write unions as disjoint unions and apply Ax 3
                            \begin{align*}
                                A = (A\cap B) \cup (A\cap B^c)  &\Rightarrow P(A) = P(A\cap B)+P(A\cap B^c) \tag{Ax 3}\\
                                                                &\Rightarrow P(A\cap B^c) = P(A) - P(A \cap B)
                            \end{align*}
                        \end{proof}
                    \item  $A\subset B \Rightarrow P(A) \leq P(B)$
                        \begin{proof}
                            \begin{align*}
                                B = A \cup (B \cap A^c) , 
                                 A \cap (B\cap A^c) =\phi\\
                                \Rightarrow P(B)    &= P(A) + P(B \cap A^c) \\
                                                    &\geq P(A) \tag{Ax 1}
                            \end{align*}
                        \end{proof}
                    \item For any two events A and B, $P(A\cup B) = P(A) +P(B) -P(A\cap B)$
                        \begin{note}
                            if $A\cap B = \phi, \text{ then } P(A\cup B )= P(A) + P(B)$
                        \end{note}
                        \begin{proof}
                            \begin{align*}
                                A \cup B &= \underbrace{(A \cap B^c) \cup (A\cap B) \cup (A^c \cap B)}_{\text{mutually exclusive}}\\
                                \Rightarrow P(A\cup B)  &= P(A\cap B^c) + P(A\cap B ) + P(A^c \cap B) \tag{Ax 3}\\
                                                        &= P(A) - P(A\cap B) +P(A\cap B) + P(A \cap B) +P(B) -P(A\cap B) \tag{Thm 3}\\
                                                        &=P(A)+P(B) -P(A\cap B)
                            \end{align*}
                        \end{proof}
                \end{enumerate}
                \begin{corollary}
                    For any event A, $0 \leq P(A) \leq 1$
                \end{corollary}
            \end{theorem}

            \begin{note}
                Do not use tree diagram to present your answer. Start by defining the simplest possible event then 
                construct more complicated event by using set operations.
                
                "Either ... or...", "At least" $\Rightarrow \bigcup$ ,\tab "and" $\Rightarrow \bigcap$,
                \tab "Not" $\Rightarrow $ complement.
                \\Then apply axiom or theorem.
            %  
            \end{note}
            
            \begin{example}
                Suppose that it's known that 20\% of people smoke and that 1\% of old people will developpe lung cancer.
                Suppose that the probability of someone will either smoke or developpe lung cancer is 0.205.
                Let A:=the event of someone smokes. \\ B:= the event of someone has cancer
                \\Then P(A) = 20\%, P(B) = 1\% and P($A\cup B$) = 0.205
                \begin{enumerate}
                    \item Find the proportion of people who smoke and developpe lung cancer.    
                        \begin{solution}
                            WTS P($A\cap B$)
                            \begin{align*}
                                P(A\cup B) &= P(A) + P(B) -P(A\cap B) \tag{thm 5}\\ 
                                \Rightarrow P(A \cap B) &=P(A) + P(B) - P(A\cup B)\\
                                                        &=0.2+0.01-0.205\\
                                                        &=0.005
                            \end{align*}
                        \end{solution}
                    \item What is the probability that someone does not smoke but have lung cancer.
                        \begin{solution}
                            WTS P($A^c \cap B$)
                            \begin{align*}
                                P(A^c \cap B)   &= P(B) - P(A\cap B) \tag{thm 3}\\
                                                &=0.01 - 0.005\\
                                                &=0.005
                            \end{align*}
                        \end{solution}
                    \item What is the probability that someone smokes but does not have lung cancer.
                        \begin{solution}
                            WTS P($A\cap B^c$)
                            \begin{align*}
                                P(A\cap B^c)&= P(A) - P(A\cap B) \tag{thm 3} \\
                                            &=0.2 - 0.005\\
                                            &=0.195
                            \end{align*}
                        \end{solution}
                    \item What is the probability that someone neither smoke nor have lung cancer. 
                        \begin{solution}
                            WTS P($A^c \cap B^c$)
                            \begin{align*}
                                P(A^c \cap B^c) &= P((A\cup B)^c) \tag{De Morgan's Law}\\
                                                &=1-P(A \cup B) \tag{thm 1}\\
                                                &=1-0.205\\
                                                &=0.195
                            \end{align*}
                        \end{solution}
                \end{enumerate}
            \end{example}
        \subsection{Tools for calculating probability}
            \begin{theorem}
                Let S be a finit sample space with N equally likely outcomes. Let E be any event in S. 
                Then $$P(E) = \frac{\mid E \mid }{N} = \frac{\text{\# of outcoms in E}}{Tot. possible outcomes}$$
                \begin{note}
                    The calculation of a probability can be then reduced to a counting problem
                \end{note}
            
            
                \begin{proof}
                    write the event E as the union of the elementary outcomes i.e 
                    \begin{align*}
                        E       &= \bigcup_{w_i \in E} w_i \Rightarrow P(E) = P(\bigcup_{w_i \in E}w_i) = \sum_{w_i\in E} P(W_i)\\
                        P(S)    &= \sum_{i=1}^N P(w_i) = 1 \tag{Ax 2} \Rightarrow P(w_i) = \frac{1}{N} \tab \forall i = 1,...,N \\
                        \text{Hence:\tab} P(E) &= \sum_{i=w_i\in E} \frac{1}{N} \\
                                                &=\frac{1}{N} \sum_{i=w_i\in E} 1 \\
                                                &=\frac{\mid E\mid}{N}
                    \end{align*}
                \end{proof}
            \end{theorem}
        \newpage   
    \section{Lec 04, Jan 17}
        we want a sample space with equally likely outcomes. 
           \begin{recall}
               S = \{1,2, ... , 10\},\tab
               R = \{1,2,3,4,5,6\}, \tab
               G = \{7,8,9,10\}
               \\All of these outcomes are reasonably equally likely. There are N=10 such outcomes.
               \\Therefore by the above thm(last class), $P(R) = \frac{\text{\# of ways to get a red marble}}{\text{tot. no. of possible outcomes}} = \frac{6}{10}$.
           \end{recall}
           
           \begin{note}
               Now although the above thm(last class) is easy to understand, the counting can sometimes be very difficult.
               it is useful to have some counting tools
           \end{note} 
        \subsection{Counting Rule}
            \begin{enumerate}
                \item If you have a set of n distinct object, then the number of ways to order the objects in n!
                \item If you have a set of n distinct object, then the number of ways to draw r object from the set, 
                     and the order is unimportant, sampling $\textbf{without}$ replacement is denoted as "n choose r". 
                     \[\binom{n}{r} = \frac{n!}{(n-r)!r!}, \tab \text{note: 0! = 1 by def}\]
                \item If we have a set of n distinct objects. The number of ways to draw r objects from these n and the order does matter, sampling 
                    $\textbf{without}$ replacement is denoted by "n permutation r" 
                    \[P(n,r) = \frac{n!}{(n-r)!}\]
                \item $\textbf{Multiplication Rule}$ (Sausage Rule)
                       \\Suppose that you have k set of $n_1, n_2, ... , n_k$ distinct objects respectively. 
                       The number of ways to form a set by selecting one object from each set is given by 
                       \[n_1*n_2*...*n_k\] 
            \end{enumerate}
        
        \subsection{The Birthday Problem}
            Suppose there are n people in a room. What is the probability that at least two have the same birthday?
            \begin{proof}\tab
                \\Suppose that there are 356 possbile birthday
                \\Let E := event that at least two people have the same birthday.
                \\It is easier to compute $P(E^c)= $ P(no two have the same birthday), then 
                \[P(E) = 1-P(E^c) \tag{by Thm 1}\]
                The sample space here is S = \{(Jan 1, ..., Jan 1), ... , (Dec 31, ..., Dec 31)\}
                \\First we assume that all of these outcomes are equally likely. There are finitely many of them.
                Therefore 
                \begin{align*}
                    P(E^c)  &= \frac{\text{\# of ways that } E^c \text{can occur}}{Tot. no. of outcomes in s}\\
                            &= \frac{\text{P(365,n)}}{365^n}\\
                            &= \frac{365*364*...*(365-n+1)}{365^n }\\
                    \intertext{Hence}
                    P(E)    &= 1 - \frac{365*364*...*(365-n+1)}{365^n }
                \end{align*}
            \end{proof}
        
        \subsection{The Fish in the Lake Problem}
            Suppose a lake has N fish in it, of which $\textbf{a}$ are tagged and $\textbf{N-a}$ are untagged.
            If you draw a sample n fish from the lake, sampling $\textbf{without}$ replacement,
            What is the probability of getting $\textbf{x}$ tagged fish in my sample?\\
            \begin{proof}\tab
                \\We want a sample space with equally likely outcomes. 
                \\Start by numbering the fish from 1 to N.
                \\We will suppose that the fish with numbers 1,...,$\textbf{a}$ correspond to those with tags and 
                the remaining $\textbf{N-a}$ numbers to the untagged fish.
                \\An outcome for our experiment is defined to a set of n integers selected from the integers 1,...,N
                \\The order is considered unimportant and assume that all sets of n numbers are equally likely.
                \\Hence we can use our thm to solve the problem. 
                \\Let E:=event that there are x tagged in sample 
                \begin{align*}
                    P(E) & = \frac{\text{number of ways to get x tagged }}{\text{total number of poss. outcomes}} \\
                    \intertext{We have }
                    \text{Tot. number of possible outcomes} &= \text{number of ways to draw n integers from a set of N distinct integers}\\
                                                            & = \binom{N}{n} \tag{counting rule 2.}    
                \end{align*}
                Now use $\textbf{Multiplication Rule}$
                \\each sub-sausage contains $x\leq a$ integers selected from integer 1,...,a  
                \\each sub-sausage contains n-x integers selected from integers (a+1),...,N.
                So we must count the number of objects in each of these two sausages, $n_1, n_2$. 
                say for the number of ways to get x tagged fish is = $n_1*x*n_2$
                we have \[n_1 = \binom{a}{x}, \tab 
                        n_2 = \binom{N-a}{n-x}\]
                finally 
                \begin{align*}
                    P(\text{x tagged fish out of n})    &= \frac{\binom{a}{x}*\binom{N-a}{n-x}}{\binom{N}{n}}\\
                \end{align*}
            \end{proof}

            \subsection{Capture \& Recapture Problem}   
                Have N fish in the lake 
                \textbf{\textit{Capture Phase}}
                    \begin{enumerate}
                        \item Remove and tag \textbf{\textit{a}} fish.
                        \item Return the fish to the lake 
                    \end{enumerate}
                \textbf{\textit{Recapture Phase}}
                    \begin{enumerate}
                        \item capture \textbf{\textit{n}} fish
                        \item Count how many tagged fish in the recaptured sample
                    \end{enumerate}
                \begin{align*}
                        P(X=x) &= \frac{\binom{a}{x}*\binom{N-a}{n-x}}{\binom{N}{n}}\\
                    \intertext{N is unknown, But}
                        \frac{a}{N} &\approx \frac{x}{n}\\
                        N &= \frac{a*n}{x}\\
                \end{align*}
                However, in real world, the captured face tends to be harder to be recaptured
    \newpage
    \section{Lec 05, Jan 22}
        \subsection{Conditional Probability}
                \textbf{\textit{Idea: }} Sometimes, knowning that an event A has occured influences the probability
                                            that the event B will occure. 
                \begin{example}
                    In our marble problem, The probability of getting a red marble on the second of two draws (\textbf{without} replacement)
                    knowning that we got a red on the first, is different from simply the probability of getting a red on the second draw.
                    Argument 
                    \begin{align*}
                        P(R_2) &= P[ (R_2 \cap G_1) \cup (R_2 \cap R_1)]\\
                                &=P[R_2 \cap G_1] + P[R_2 \cap R_1]
                    \end{align*}
                    which is easy to see, $P(R_2)$ is different from $P(R_2 \text{ knowning } R_1)$
                    %\begin{align*}
                    %    P(R_2 \text{ knowning } R_1) &= 
                    %\end{align*}
                \end{example}
                We therefore feel justified in formally defining the notion of "Conditional Probability"

                \begin{definition}
                    Let A and B be two events such that P(A) $\neq $ 0, then we define the probability of B given A as follows:
                    \begin{align*}
                        P[B \text{ given } A] := P[B \mid A] = \frac{P[A\cap B]}{P(A)}
                    \end{align*}
                    \textbf{\textit{Note: }}the RHS is the ratio of two probability and we have defined probability (The 3 Axioms)
                \end{definition}
                \begin{note} \tab 
                    \begin{enumerate}
                        \item we need to check whether conditional probability satisfies the 3 Axioms.
                            \begin{enumerate}
                                \item $P(B \mid A) \geq 0$ 
                                    \begin{proof}
                                        \[P[ B \mid A] = \frac{\underbrace{P[A \cap B]}_\geq 0}{\underbrace{P(A)}_\geq )}\]
                                        Hence true
                                    \end{proof}
                                \item $P[S \mid A ] =1 $
                                    \begin{proof}
                                        \[P[ S \mid A] = \frac{P[A \cap S]}{P(A)} = \frac{P(A)}{P(A)} = 1\]
                                        Hence true
                                    \end{proof}
                                \item $P\big[\bigcup B_i \mid A \big] = \sum_{i=1}^{\infty} P[B_i \mid A] \tab where B_i \cap B_j = \phi \tab \forall i \neq j$
                                    \begin{proof}
                                        (exercise)
                                    \end{proof}
                            \end{enumerate}
                        It then follows that the 5 theorem also go through for conditional Probability
                        \item $P(A \mid B) = \frac{P(A \cap B)}{P(B)} \tab \text{ if } P(B) \neq 0$
                            
                    \end{enumerate}
                \end{note}
                The definition of conditional probability leads to a fundamental theorem that allows us to sometimes
                find the probability of an intersection 

        \subsection{Multiplication Rule for Conditional Probability}
            Follows immediately from the definition of conditional probability
            \begin{align*}
                P(B \cap A) &= P(B \mid A)* P(A )\\
                            &=P(A \mid B) * P(B)
            \end{align*} 
            The hope is that when you are required to find $P[A\cap B]$, you know either P(A) or P(B) 
            and one of the conditional probability. 
            \\In word problems, 
            \begin{align*}
                \text{"of those that" } \Rightarrow \text{ Conditional Probability}
            \end{align*}
            Do not confuse "and" with "given that"

            \begin{example}
                We have two inspectors for items coming off an assembly . The proportion of items that are declared non-defective by the first inspector is 0.90
                of those items that are declared non-defective by inspector 1, 0.95 are declared non-defective by inspector 2. 
                What is the probability that an item is declared non-defective by both inspectors.
            \end{example}

            \begin{solution} \tab 
                \\
                \\Let $ND_i $ (i=1,2) := event non-defective for each of the inspectors resp.\\
                \textbf{WTS:} $P\big[ND_1 \cap ND_2 \big]$ \\
                Given: $P[ND_1] = 0.90$ and $P\big[ND_2 \mid ND_1 \big] = 0.95$ \\
                \begin{align*}
                    P[ND_2 \cap ND_2] &= P[ND_2 \mid ND_1]* P[ND_1] \\
                                        & = 0.90 * 0.95 \\
                \tab \tab \tab \square
                \end{align*}
            \end{solution}
            \textbf{*Extension: } LEt $A_1 , ..., A_n$ be any sequence of events. Then 
            \begin{align*}
                P[A_1 \cap ... \cap A_n] &= P[A_n \mid A_1 \cap ... \cap A_{n-1}]*P[A_1\cap ... \cap A_{n-1}]\\
                                            &= P[A_n \mid A_1 \cap ... \cap A_{n-1}] * P[A_{n-1} \mid A_1 \cap ... \cap A_{n-2}]*P[A_1\cap ... \cap A_{n-2}]\\
                                            &= ...\\
                                            &= \Pi_{i=1}^n P[A_i \mid  A_1 \cap ... \cap A_{i-1}]*P(A_0) \tag{i=1,...,n}
            \end{align*}
            \begin{example}\tab \\
                \begin{align*}
                    P (A_1 \cap A_2 \cap A_3)   &= P(A_3 | A_1 \cap A_2) P(A_1 \cap A_2) \\
                                                &= P(A_3 | A_1 \cap A_2) P(A_2 |A_1) P(A_1) \\
                \end{align*}
            \end{example}
            The process of repeatedly conditioning stating with the last event is called the process of \textbf{\textit{conditioning backwards}}.\\
            When you are required to find the probability of the intersection of several events, think of conditioning backwards
    \newpage
    \section{Lec 06, Jan 24}
            \subsection{Conditioning Backwards}
                \begin{definition}
                    The process of repeatedly conditioning stating with the last event is called the process of \textbf{\textit{conditioning backwards}}.\\
                    When you are required to find the probability of the intersection of several events, think of conditioning backwards
                \end{definition}
                
                \begin{example}
                    \textbf{The Marble Problem} \\
                    \begin{enumerate}
                        \item Suppose that you draw 2 marbles \textbf{without} replacement. What is the probability that the second marble drawn is green?
                            \begin{solution} Conditioning Backwards. \\
                                \begin{align*}
                                    G_2     &= (R_1 \cap G_2) \cup (G_1 \cap G_2) \\
                                    \intertext{\tab Implies}
                                    P(G_2)  &= P((R_1 \cap G_2) ) + P((G_1 \cap G_2) ) \tag{Ax 3} \\
                                            &= P(G_2|R_1)P(R_1) + P(G_2|G_1) P(G_1) \\
                                            &= \frac{4}{9} * \frac{6}{10} + \frac{3}{9}* \frac{4}{10}
                                \end{align*}
                            \end{solution}
                        \item Suppose that you draw 5 marbles. What is the probability that you will get the sequence $R_1, R_2, G_3, G_4, R_5$.
                            \begin{solution} Conditioning Backwards
                                \begin{multline*}
                                    P(R_1 \cap R_2 \cap G_3 \cap G_4 \cap R_5)  = P (R_5 | R_1 \cap R_2 \cap G_3 \cap G_4 ) 
                                                                                    * P(G_4 |R_1 \cap R_2 \cap G_3 )\\
                                                                                        * (P(G_3|R_1 \cap R_2)
                                                                                            *(P(R_2 |R_1)*P(R_1)\\
                                \end{multline*}
                                Hence 
                                \begin{align*}
                                    P(R_1 \cap R_2 \cap G_3 \cap G_4 \cap R_5)  &= \frac{4}{6}* \frac{3}{7}* \frac{4}{8}* \frac{5}{9} *\frac{6}{10}
                                \end{align*}
                            \end{solution}
                    \end{enumerate}
                \end{example}
                The following theorem on conditional probability are fundamental 
            \subsection{The Law of Total Probability}
                \begin{theorem}\tab \\
                    Let A be any event, let $B_1, B_2, ...$ be m events that satisfy the following 
                    \begin{enumerate}
                        \item $B_i \cap B_j = \phi \tab \forall i \neq j$ 
                        \item $\bigcup\limits_{i=1}^m B_i = S$ we call \{$B_1,B_2,...$\} a partition of S
                    \end{enumerate}
                    Then $P(A) = \sum\limits_{i=1}^m P(A |B_i) P(B_i)$ 
                \end{theorem}
                \begin{proof}
                    (Of theorem)
                    \\known $A = \bigcup\limits_{i=1}^m \underbrace{(A\cap B_i)}_{\text{all disjoint}}$
                    Hence 
                    \begin{align*}
                        P(A)    &= \sum_{i=1}^m P(A \cap B_i) \tag{Ax 3} \\
                                &= \sum_{i=1}^m P(A | B_i) P(B_i)
                    \end{align*}
                \end{proof}
                \begin{note}
                    Maybe A is complicated and it is difficult to find its probability directly 
                    or the given information does not provide P(A) directly. \\
                    The hope is that we can find P(A$| B_i$) easily or that they come with the provided information 
                    and that we know P($B_i$).\\
                    In word problem, the clue to use the Law of Total Probability is that you are given a bunch of conditional probability and the probability $P(B_i)$ and you are asked to find P(A).
                \end{note}
        
            \subsection{Baye's Theorem}
                \begin{theorem}
                    Let A and $B_1, B_2,...$ be defined exactly as in the \textit{Law of Probability}. 
                    Then we can write  $$P(B_k | A) = \frac{P(A|B_k) *P(B_k)}{\sum_{i=1}^m P(A|B_i) P(B_i)} \tab (k=1,2,...,m)$$
                \end{theorem}
                \begin{proof}
                    \begin{align*}
                        P(B_k|A)    &= \frac{P(B_k \cap A)}{P(A)} \\
                                    &= \frac{\overbrace{P(A|B_k)P(B_k)}^\text{mult. rule}}{\underbrace{\sum_{i=1}^m P(A|B_1)P(B_i)}_{\text{Law of tot.Prob.}}}
                    \end{align*}
                \end{proof}
                \begin{note}
                    Mathematically, Baye's Theorem allows you to reverse one or more given conditional probability. \\
                    In word problem, the clue to use the Baye's thm is that you are required to reverse one or more conditional probability statement. 
                \end{note}
                \newpage
                \begin{example}
                    Suppose that there is a diagnostic test for breast cancer and that  in a certain population $\frac{5}{1000}$ women have breast cancer.
                    Known that the test has the following properties:
                    \begin{enumerate}
                        \item if a woman has breast cancer, the test will be positive 95\% of the time.
                        \item if a woman does not have breast cancer, the test will be negative 95\% of the time.
                    \end{enumerate}
                    Question is 
                    \begin{enumerate}
                        \item[i)] What proportion of women will test positive?
                            \begin{solution} \tab\\
                                \begin{align*}
                                    \text{Let Pos }&:=\text{ event that a test is positive } \\
                                    \text{Let Neg }&:=\text{ event that a test is negative } \\
                                    \text{Let Bc }&:=\text{ event that a woman has breast cancer} \\
                                    \text{Let Bc$^c$ }&:=\text{ event that a woman does not have breast cancer} \\
                                \end{align*}
                                Known
                                \begin{align*}
                                    P(\text{Pos } | \text{Bc})   &= 95\% \\
                                    P(\text{Pos } | \text{Bc}^c)   &= 1-95\%  = 0.05\\
                                    P(\text{Neg } | \text{Bc}^c) &= 95\% \\
                                    P(\text{BC})                 &= 0.005 \\
                                    P(\text{BC}^c)               &= 1-0.005\\
                                \end{align*}
                                We have that BC and BC$^c$ are disjoint and BC $\cup  BC^c $ = S. \\
                                Therefore by the Law of Tot. Prob. 
                                \begin{align*}
                                    P(\text{ Pos }) &= P(\text{Pos } | \text{Bc}) * P(\text{BC}) + P(P(\text{Pos } | \text{Bc}^c) )* P(\text{Bc}^c))\\
                                                    &= 0.95*0.005 + 0.05 *(1-0.005) \\
                                                    &= 0.054
                                \end{align*}
                            \end{solution}
                        \item[ii)] If a woman tests postive, what is the probability that she has breast cancer?
                            \begin{solution}
                                WTS P(Bc $|$ Pos).\\ \textbf{note: } we are required to reverse the  P(Pos $|$ Bc) (Baye's) 
                                \begin{align*}
                                    P(\text{Bc }| \text{Pos })  &= \frac{P(\text{Pos } | \text{Bc})* P(Bc) }{P(\text{Pos } | \text{Bc})* P(Bc)+ P(\text{Pos } | \text{Bc}^c)* P(Bc^c)}\\
                                                                &= \frac{0.95*0.005}{\underbrace{0.054}_{\text{from part i)}}} \\
                                                                &= 0.087
                                \end{align*}
                                Some comments on Baye's Theorem and diagnostic tests: \\
                                    In the world of diagnostic tests P(Pos $|$ Disease) is called the sensitivity of the test
                                    P(neg $| \text{Disease}^c$) is called the specificity.
                                    P(Disease) is called the disease prevelance and 
                                    P(Disease $|$ Pos) is called the postive predictive value of the test.
                                    Note that pos predictive value depends on the sensitivity, specitivity and prevalue of the diseasea. 
                            \end{solution} 
                    \end{enumerate}
                \end{example}

    \newpage
    \section{Lec 07, Jan 29}
        Some comments on Baye's Theorem and diagnostic tests: \\
        In the world of diagnostic tests \textbf{P(Pos $|$ Disease)}:= the sensitivity of the test\\
        \textbf{ P(neg $| \text{Disease}^c$)}:= the specificity.\\
        \textbf{P(Disease)}:= the disease prevelance and\\ 
        \textbf{P(Disease $|$ Pos)}:= the postive predictive value of the test.\\
        Note that pos predictive value depends on the sensitivity, specitivity and pre. value of the diseasea. \\
        
        \subsection{Statistical Independence*}
            \textbf{Idea}: sometimes the occurance of an event A 
                            does \textbf{NOT} influence the probability that an event B will occur.
            \begin{example}
                In the marble problem, supose that you draw two marbles \textbf{with} replacement. 
                What is the probability that the second marble is drawn is red given the first is green. 
                (i.e what is P[$R_2 | G_1$]). \\
                Clearly, we obtained a green on the first draw has no impact on the probability of a red on the second draw, 
                since the box was returned to its original composition.
                This idea leads to a definition of the independence of two event A and B
            \end{example}
            
            \begin{definition}
                \tab
                \begin{enumerate}
                    \item  We say that the events A and B are independent $\iff $ P[ $B | A$ ] = P(B) \\
                            \textbf{Note:} This definition is while intuitive, it is not easily extended to the notion of independence of more than two events.
                    \item (non-intuitive but extendable) The events A and B are said to be independent, 
                            \\$\iff $ $P(A \cap B) = P(A)*P(B)$.
                \end{enumerate}
            \end{definition}

            \begin{theorem}
                A and B are independent according to DEF 1 $\iff $ they are independent according to DEF 2.
                (i.e The above two definition are equivalent.)
            \end{theorem}
            \begin{proof}\tab 
                \begin{description}
                    \item[($\Rightarrow$)]
                    Assume A and B are independent according to DEF 1, WTS they are also independent according to DEF2.\\
                    Then P(B $|$ A ) = P(B).\\
                    Hence by the Multiplication rule 
                    \begin{align*}
                        P(A \cap B) &= P(B | A)*P(A) \\
                                    &= P(B)*P(A)
                    \end{align*}
                    \item[($\Leftarrow$)]
                        Trivial
                \end{description}
            \end{proof}
            
            \begin{definition}
                The events $A_1, A_2,..., A_n$ are sait to be mutually independent if 
                P\big[$A_{i_1} \cap ... \cap A_{i_k}$] = $P[A_{i_1} \big]*...*P[A_{i_k}]$ for all subset of 
                $A_{i_1}...A_{i_k}$ selected from $A_1$ to $A_n$. 
            \end{definition}
            \begin{example} \tab
                \begin{enumerate}
                    \item $P(A\cap B \cap C) = P(A)P(B)P(C)$
                    \item $P(A \cap B ) = P(A) P(B)$ etc...
                \end{enumerate}
            \end{example}

            We also define independence for an infinite sequence of events $A_1,A_2,...A_3$
            \begin{definition} 
                    The event $A_1, A_2,...$ are independent if and only if every finite set of $A_i$ is independent according to the previous definition of independence.  
            \end{definition}

            \begin{note}\tab 
                \begin{enumerate}
                    \item Events $A_1,A_2,...,A_n$ are said to be pairwise independent if
                            $P(A_i \cap A_j) = P(A_i)P(A_j)$ for all $i\neq j$. \\
                            It can be shown that pairwise independence does \textbf{NOT} imply mutual independece. 
                    \item we write $A \independent B$ to denote A is independent of B
                    \item If $A \independent B $ then $B \independent A$ and vice-versa. \\
                            Further, we have if A, B and C are mutually independent then 
                            \begin{enumerate}
                                \item $A^c \independent B $
                                \item $(A \cup B )^c \independent C$
                                \item $(A^c \cap B^c )\independent C$
                                \item $(A \cup C)^c \independent B$
                                \item $A^c \independent B^c$ etc...
                                        \begin{proof}
                                            WTS P($A^c \cap B^c$) = P($A^c$)P($B^c$) 
                                            \begin{align*}
                                                P(A^c \cap B^c) &= P((A\cup B)^c) \tag{De Morgan} \\
                                                                &= 1 - P(A \cup B) \tag{Thm 1}\\
                                                                &= 1- \Big[ P(A) + P(B) - P(A \cap B)\Big] \tag{Thm 5} \\
                                                                &= 1- P(A) - P(B) + P(A \cap B) \\
                                                \intertext{Now notice}
                                                P(A^c)P(B^c)    &= (1-P(A))(1-P(B))\\
                                                                &= 1-P(A)-P(B) +P(A)P(B) \\
                                                \intertext{Hence}
                                                P(A^c \cap B^c) &= P(A^c)P(B^c)
                                            \end{align*}
                                        \end{proof}
                            \end{enumerate}
                    \item \underline{In fact the following is true}\\
                    Suppose that you have two sets $A_1,A_2,...A_n$ and $B_1,B_2,..., B_n$. 
                    We say that the set of A is independent of set B if the probability of the intersection of every set of A with the intersection of every set of B 
                    is the product of the intersection of the set of A and the set of B.\\
                    E.g $P\Big[(A_3 \cap A_5 \cap A_6 ) \cap (B_1 \cap B_2)\Big] = P\Big[A_3 \cap A_5 \cap A_6\Big]* P\Big[B_1 \cap B_2\Big]$
                \end{enumerate}
            \end{note}
            \newpage
    
    \section{Lec 08, Jan 31}
        \subsection{The Role of Independence}
            \begin{enumerate}
                \item The most important Role of Independence in probability is the following:\\
                        If you can assume independece base on your knowledge of the substantive area 
                        and/or the way the experiment was carried out.
                        Then subsequent probability calculations often become a lot easier than if you cannot make this assumptions.\\
                        This is so since the probability of intersection. becomes product of probabilities 
                        rather than product of conditional probabilities requiring knowledge of these conditional probabilities.
                \item Second, we may want to decide whether evetns are independent (i.e This may be the goal.)\\
                        For instance, one may wish to know, whether recovery time from abdominal surgery is independent of the temperature of the operating room.
                \item The relation between disjointness and independence: \\
                        It turns out that these notions are completely different.
                        Disjointness is entirely a set property whereas independece depends on how probabilities are assigned to these events.
                        The following theorem says it all:
                        \begin{theorem}
                            Suppose that A and B are disjoint, then A and B are independent only if either P(A) = 0, or P(B) = 0.
                        \end{theorem}
                        \begin{proof}
                            \begin{align*}
                                \text{A, B disjoint}    &\Rightarrow A \cap B = \phi\\
                                                        &\Rightarrow P(A \cap B) = P(\phi) = 0\\
                                \intertext{Now if A $\independent$ B, then we must have: }
                                P(A \cap B) = P(A) P(B) &\Rightarrow P(A) P(B) = 0 \\
                                                        &\Rightarrow P(A) = 0, \text{ or } P(B) = 0\\
                            \end{align*}
                        \end{proof}
                        \begin{note}
                            sometimes you will be required to find $P(A \cup B)$, you have from theorem 5, that 
                            $P(A \cup B) = P(A) + P(B) - P(A \cap B)$, \\
                            To deal with $P(A \cap B)$:
                            \begin{enumerate}
                                \item $A \cap B = \phi \Rightarrow P(A \cap B) = 0$
                                \item $A \independent B \Rightarrow P(A \cap B) = P(A)P(B)$
                                \item A and B are dependent, then $P(A \cap B) = P(B \mid A) P(A) $ (note, this is always true)
                            \end{enumerate}
                        \end{note}
            \end{enumerate}

            \newpage
            \begin{example} (on how independence can be used)
                \begin{note} Preliminary note
                    \begin{enumerate}
                        \item when sampling \textbf{without} replacement, the outcomes in the sequence of draws are dependent.\\
                                Thus if you have a box of 10 items if which 4 are defective and you remove 2 without replacement,
                                whether or not you observe a defective on the second draw will depend on what was removed on the first draw.
                                However, if "the box" from which we sample is very large, relative to the size of the sample,
                                we may regard the outcomes of our draws, as being roughly independent.
                    \end{enumerate}
                \end{note}
                Suppose that in a very large city, 20\% of people have a certain genetic mutation. If 10 people are examined \\
                Clearly we are sampling without replacement (by design). \\
                Let the outcome on trial i, be $M_i$:= there is a mutation for subjects i. \\
                $M_i^c$:= there is no mutation for subjects i. \\
                Let X:= the number of mutations in these ten trials. (randome variable.)
                \begin{enumerate}
                    \item What is the probability that exactly 2 will have the mutation?
                        \begin{solution} 
                            WTS P(X = 2). Proceed as following
                            \begin{enumerate}
                                \item Find the probability of a specific configuration of mutations and non-mutations in 10 trials that result in 2 mutations out of 10.\\
                                        Consider the configuration 
                                        \begin{align*}
                                            (M_1, M_2, M_3^c,...., M_{10}^c)    &= M_1 \cap M_2 \cap M_3^c \cap ... \cap M_{10}^c \\
                                            P(M_1, M_2, M_3^c,...., M_{10}^c)   &= P(M_1)P(M_2)P(M_3^c)...P(M_{10}^c) \tag{Rough independenc (large 'city' - small sample)} \\
                                                                                &= 0.2* 0.2* 0.8*...*0.8
                                        \end{align*}
                                        Now notice, all configurations which result in exactly two mutations will have probability $(0.2)^2(0.8)^8$
                                \item Sum up the probability of all such configurations
                                        \begin{align*}
                                            P[X=2]  &= P \Big[\bigcup_{\text{k=1}}^{\text{all config.}} \text{ configuration k with 2 mutations }\Big] \\
                                                    &=\sum_{\text{k=1}}^{\text{all config.}} P \Big[ \text{ configuration k with 2 mutations }\Big] \tag{Ax 3. (config. disjoint)}\\
                                                    &= (0.2)^2(0.8)^8 \sum_{\text{k=1}}^{\text{all config.}} 1 \\
                                                    &= (0.2)^2(0.8)^8 \binom{10}{2}
                                        \end{align*}
                            \end{enumerate}
                            
                        \end{solution}
                        \newpage
                    \item What is the probability that at least 2 will have the mutation?\\
                        \begin{solution}
                            WTS P[x $\geq 2$] \\
                            \begin{align*}
                                P [ x \geq 2]   &= \sum^{10}_{k=2} P[x = k] \\
                                                &= \sum^{10}_{k=2} (0.2)^k (1-0.2)^{10-k} \binom{10}{k} \\
                                \intertext{or completationally simpler:}
                                                &= 1 - P[x < 2]\\
                                                &= 1 - \big[P [x=0] + P[x = 1]\big]\\
                                                &= 1 - \Bigg[\binom{10}{0}(0.2)^0(0.8)^{10-0} + \binom{10}{1}(0.2)^1(0.8)^{10-1}\Bigg]\\
                            \end{align*}
                        \end{solution}
                        \begin{note}
                            If the city has 10 million people with 20\% having the multation then the exact answer will be 
                            $$P[x=2] = \frac{\binom{2 \text{million}}{2} \binom{8 \text{million}}{8}}{\binom{10 \text{million}}{10}}$$
                            If you are not giving the actual size of the city, but told it's large. 
                            You have to use the independent approximation.
                        \end{note}
                \end{enumerate} 
            \end{example}
            \newpage

    \section{Lec 09, Feb 05}
        \subsection{Random Variable}
            \textbf{Idea:} Often not interested in the outcome of a random experiment but rather in real numbers
                            that we can associate with each of these outcomes.
            \begin{example}
                A pathologist looking a grid on slide, where each rectangle contains either a red blood cell or a white blood cell. 
                The pathologist is not really interested in the SEQ $R_1, R_2, R_3, W_4, R_5, W_6 ...$ 
                rather she is interested in the number of red/white cells on the slide. 
                This idea lead to the definition of the random variables that assigns a real number to each possible outcome of the experiment.
            \end{example}

            \begin{definition}
                We call the function X(w) that assigns a real number of X(w) to every elementary outcome $\omega \in E$ a real value random variable. 
                That is $X: w \rightarrow \Re$
            \end{definition}
            \begin{note} \tab 
                \begin{enumerate}
                    \item X is just a function. However, the argument and random outcomes of an experiment. \\
                            i.e The outcomesare uncertain or random BEFORE the experiment is carried out. 
                            Hence, the value of the function X are uncertain or random prior to the experiment. 
                            This is why we called the function, X, a random variable.
                    \item Always denote random variable by capital letters (e.g X, Y, Z etc)
                    \item Once the experiment has been carried out and an $\omega \in S$ has been observed X(w) is a real number and 
                            it's no longer random. i.e a realized or post experiment value of a random value is not random.\\
                \end{enumerate}
            \end{note}
            \begin{definition}
                Random variable are broadly classified into one of 2 types
                \begin{enumerate}
                    \item \textbf{Discrete R.V} \\
                            By definition, a random variable is said to discrete if it can assume at most a countably infinite number of distinct value.
                            \begin{example}\tab
                                \begin{enumerate}
                                    \item The number of times we get heads in 3 tosses of a coin (here could assume the values of X to be 0, 1, 2, 3)
                                    \item The times between arrivals of a bus we assume to the nearest minute.
                                    \item The number of explorality  wells that are drilled before oil is first struck (here, could assume the value of X to be 1,2,3...)
                                \end{enumerate}
                            \end{example}
                    \item \textbf{Countinous R.V} \\
                            There are random variables that can be assumed any real value in some interval. 
                            \begin{example}\tab
                                \begin{enumerate}
                                    \item The $\textbf{\underline{exact}}$ time between the arrival of buses at a bus stop
                                    \item The $\textbf{\underline{exact}}$ time to recovery following surgery.
                                    \item The $\textbf{\underline{exact}}$ blood pressure of a patient.
                                \end{enumerate}
                            \end{example}
                \end{enumerate}
            \end{definition}
        \subsubsection{Specification of R.V}
            \begin{enumerate}
                \item In classical mathematics we specify a function by giving its domain and the values of the function at these value of X.
                        Since a R.V X has a inherent uncertainties (prior to the experiment) when specify X, we need to specify the probability that X will assume its various values(the rough idea)
                \item The following is one way to specify a R.V in the case where R.V X is either discrete or continous
            \end{enumerate}

            \begin{definition}
                Let X be a random variable. Let the real value function denoted by $F_x$ is defined as follows: 
                $$F_x(X) \stackrel{\text{def.}}{=} P[X \leq x] \tab \forall -\infty < x < +\infty$$
                F := the cumulative distribution function (cdf) or distribution function (df) of the random variable X. 
            \end{definition}
            \begin{note}\tab
                \begin{enumerate}
                    \item when we write $P[X \leq x]$ we mean $P[\omega: X(\omega) \leq x] = P\big[X^{-1}(-\infty, x]\big]$ \\
                            i.e $P[X \leq x]$ = P (all w in S that are mapped by X into the interval ($-\infty, x$] )
                    \item When we have specified the cdf of a random variable X it turns out that we have (in theory) uniquely specified $P[x \in A]$ for any A a subsect in the real line. 
                            i.e we can uniquely specify the probability of all events associated with X once we know the cdf $F_x$ that gives us only the probability of special types of sets, $(-\infty,x]$
                    \item We call the set of probability $\big\{P[x \in A]: A \subset \Re \text{ 7y7is an event} \big\}$ The probability distribution of Random variable X.
                \end{enumerate}
            \end{note}
            \newpage

    \section{Lec 10, Feb 07}
        \subsection{Cumulative Distribution Function}
            \textbf{Summary: }
            \begin{enumerate}
                \item The probability distribution of a random variable 
                        is the specification of the probability that $x \in A$ for every subset A of the real line.\\ 
                        i.e this is \{P(x $\in$ A): A is a subset of $\mathbb{R}$ \}
                \item Amazingly, in order to uniquely specify the probability of of all subsets, 
                        it's enough to specify the probabilty of just a small collection of subsets.\\
                        i.e it's enough to specify the probability of intervals of the type (-$\infty, x$)
                \item The theorem telling us this = caratheodory extension theorem.
                \item The real-value function of X, which gives P($X \leq x$) = P[X $\in (-\infty,x]$] is called the cumulative distribution function (cdf) and is denoted by $F_x()$
                        Thus $F_x(X) \stackrel{\textbf{def}}{=} P(X \leq x)$
                \item Thus the cumulative distribution function uniquely determines the probability distribution of a random variables.
            \end{enumerate}

            \begin{note} \tab
                \begin{enumerate}
                    \item \textbf{The CDF's Properties:} 
                            \begin{enumerate}
                                \item $F_x(X)$ is non-decreasing of a function of X. i.e if x $<$ y then $F_x(X) \leq F_x(Y)$ 
                                \item $F_x(X)$ is continuous from the right i.e $\lim\limits_{y \neq x} F_x(Y) = F_x(X)$ (basically $F_x(X)$ does not jump as you approach x from above)
                                \item $\lim\limits_{x \rightarrow \infty} F_x(X) = 1$ and $\lim\limits_{x \rightarrow -\infty} F_x(X) = 0$ in brief $F_x(\infty) = 1$ and $F_x(-\infty) = 0$ 
                            \end{enumerate}
                    \item If $F_x$ is continuous both from the left and right (i.e F is "plain" continuous, we say the random variable X is continuous)
                    \item We have P($a \leq X \leq b$) = $F_x(b) - F_x(a)$ for all a $<$ b
                    \item Here are some typical cdf's:\\ \includegraphics[scale = 0.3]{L9_1.png} \includegraphics[scale = 0.3]{L9_2.png} 
                    \item  So far, our discussion hasa applied to random variables in general. 
                            We now examine \textbf{discrete} random variables and \textbf{continuous} random variables separately
                                \begin{itemize}
                                    \item A random variable is discrete if its cdf is a step function with jumps of values $X_1, X_2,...$ which can be assumed by the random vairable X. 
                                            Thus a discrete random vairable has cdf with at most countably many jumps.
                                    \item The size of each jump at $x_0$ is obtained as $$F_X(x_0) - \lim_{y\rightarrow x_0^-} F_X(y) = F_X(x_0) - F_X(x_0^-)$$
                                            Think roughly, that the size of the jump at $x_0$ is $ F_X(x_0) - F_X(y)$ has a y \textbf{very} close to $x_0^-$
                                    \item $P[X = x_0] = F_X(x_0) = F_X(x_0^-)$ = size of the jum in $F_X$ at $x_0 $ \\
                                            If $x_0$ is not one of the value of X, then the size of the jump will be 0. This correspond to P[X = $x_0$] = 0 for such $x_0$\\
                                            If $x_0$ coincides with one of the $x_i$ then the size of the jump at that $x_i$ gives P[X = $x_i$]\\
                                            There is apart from the cdf another way to specify the probability distribution of a discrete random variable. 
                                \end{itemize}
                \end{enumerate}

            \end{note}
        \subsection{Discrete R.V and CDF}
            \begin{definition}
                If X is a \textbf{discrete} random variable, then the real valued function of X, defined as follows, is called the \textbf{probablity function} (probability mass function of the random variable)
                \begin{itemize}
                    \item Let $P_X(x) = P[X = x]$ for all x in range of X \\
                            Then we will call $P_X(x)$ the probablity function of the random variable X \\
                            i.e $P_X(x)$ gives the probability of all x. we usually specify $P_X(x)$ only for those x for wich the probablity is $>$ 0 and it's assumed that all other x have prob. = 0
                \end{itemize}
            \end{definition}
            Now it is easy to show that $P_X()$ uniquely determines the probablity distribution of a discrete random variable.
            The following theorem tells us there is 1-1 correspondence between the probability function $P_X$ and the cdf $F_X$ of a discrete random variable. \\
            Hence $P_X$ uniquely determines the probability distribution of X

            \begin{theorem} \tab 
                \begin{enumerate}
                    \item $P_X$ determines $F_X$ 
                    \item $F_X$ determines $P_X$
                \end{enumerate}
                \begin{proof}
                    \begin{enumerate}
                        \item Let $P_X$ be given, then 
                                \begin{align*}
                                    F_X(x_0)    &= P[X \leq x_0]  \\
                                                &= \sum_{\text{all x} \leq x_0 } P[X = x]  \tag{Ax 3} \\
                                                &= \sum_{x \leq x_0} P_X(x) 
                                \end{align*}
                                i.e $P_x$ determines $F_X(x_0)$ for all $x_0$
                        \item suupose $F_X$ is given \\
                                Let $X_0$ be some arbitrary value in the range of X.\\
                                Then we have 
                                \begin{align*}
                                    P_X(x_0)    &= F_X(x_0) - F_X(x_0^-)  \\
                                                &= F_X(x_0) - \lim_{y \rightarrow x_0^-} F_X(y)
                                \end{align*}
                    \end{enumerate} 
                \end{proof}
            \end{theorem}
            \newpage
            \begin{example}
                Suppose that if a flight is cancelled the airline loses 5000\$, \\
                if the flight leaves late (more than 1/2 hours late) it loses 2000\$,\\
                if it leaves on time, it makes a profit of 10000\$\\
                if it leaves late but less than 1/2 hour, it still makes 10000\$\\
                P[C] = 0.05\\
                P[$>$ 1/2 hour late] = 0.1 \\
                P[on time ] = 0.7\\
                The questions are:
                \begin{enumerate}
                    \item Find the probability function of the random variable that represent the gain on a flight
                    \item Find the cdf of this random variable.
                \end{enumerate}
            \end{example}
            \begin{solution}
                The elementary outcomes are: \\
                \tab $\omega_1 = $ cancelled. \\
                \tab $\omega_2 = $ 1/2 hour late. \\
                \tab $\omega_3 = $ on time. \\
                \tab $\omega_4 = $ less than 1/2 hour late. \\
                Let X = the gain for a flight.\\
                We have: \\
                \tab X($\omega_1$) = -5000\\
                \tab X($\omega_2$) = -2000\\
                \tab X($\omega_3$) = 10000\\
                \tab X($\omega_4$) = 10000\\
                \begin{enumerate}
                    \item We want$P_X(x) = P[X = x] $ for all x that X can assume
                            \begin{align*}
                                P[X = -5000]    &= P[\omega : X(\omega) = -5000]\\
                                                &= P[C] = 0.05\\
                                P[X = -2000]    &= P[\omega : X(\omega) = -2000]\\
                                                &= P[> 1/2 \text{ late }] \\
                                                &= 0.1\\
                                P[X = 10000]    &= P[\omega : X(\omega) = 10000]\\
                                                &= P[\text{ on time }] + P[\leq 1/2 \text{ late }]\\
                                                &= 0.7 + 1 - P[> 1/2 \text{ late }]- P[\text{ on time }] - P[C]\\
                                                &= 0.7 + 1 - 0.05 - 0.1 - 0.7 
                                                = 0.85
                            \end{align*}
                            Hence $P_X(x)$ is specified and \\
                            \textbf{PLOT:}\\ 
                            \includegraphics[scale=0.3]{L11_1.png}
                    \item To find the cdf $F_X(x)$, we need to give P[$X \leq x$] for all $-\infty < x < \infty$. 
                            \begin{align*}
                                F_X(x) &= 0                \tag{for x $<$ -5000 }\\
                                F_X(x) &= P[X = -5000] = 0.05              \tag{for -5000 $\leq$ x $\leq$ -2000 } \\
                                F_X(x) &= P[-\infty < X < -5000] + P[-5000 \leq X < -2000] \\
                                       & \tab + P[-2000 \leq X < 10000] \\
                                       &= 0 + 0.05 + 0.1 \\
                                       &= 0.15 \tag{for -2000 $\leq$ x $\leq$ 10000 } \\
                                F_X(x) &= 0 + 0.05 + 0.15 + 0.85 = 1 \tag{for 10000 $\leq$ x $< \infty$}
                            \end{align*}
                            \textbf{PLOT}: \\\includegraphics[scale = 0.3]{L11_2.png}
                \end{enumerate}
            \end{solution}
   
    \newpage
    \section{Lec 11, Feb 12}
        \subsection{Some named discrete distribution}
            Some discrete distribution arise so frequently that they are given special names. Here are some of them:
            \subsubsection{The discrete uniform distribution}
                \begin{definition}
                    The random variable X is said to be a discrete uniform distribution on the real numbers $a_1, a_2, ... ,a_N$ 
                    if $P_X[x] = P[X = x] = 1/N$ for all x = $a_1, a_2, ... ,a_N$  
                    (i.e $P_X(a_i) = 1/N$ for all i = 1, 2, ..., N)
                \end{definition}
                \begin{note} \tab
                    \begin{enumerate}
                        \item A probability function $P_X(x)$ must satisfy the following 2 conditions:
                                \begin{enumerate}
                                    \item $P_X(x) \geq 0 $ \tab \tab $\forall x$
                                    \item $\sum\limits_{\text{all x}} P_X(x) = 1$ \tab (P(S) = 1)
                                \end{enumerate}
                        \item Reason for the name, "uniform": \\
                                All possible values of X are equally probable, with probabilty 1/N
                        \item The $a_i$ need not be positive or integer valued.\\ i.e \\\includegraphics[scale = 0.3]{L11_3.png}
                        \item The discrete uniform distribution is used to model what we understand to be "complete randomness"
                    \end{enumerate}
                \end{note}

            \subsubsection{The Bernouilli Distribution}
                \begin{definition}
                    The random variable X is said to have a Bernouilli distribution with parameter p,
                    if \begin{enumerate}
                        \item $P_X(x) = p$ when x = 1
                        \item $P_X(x) = 1-p = q$ when x = 0 
                    \end{enumerate}
                \end{definition}
                \begin{note} \tab
                    \begin{enumerate}
                        \item We can write the probability function compactly as $P_X(x) = p^x(1-p)^{1-x}$ for x =0, 1
                        \item The Bernoullie Distribution is mostly used as a building block for random variables that can be regarded as sums.
                    \end{enumerate}
                \end{note}

            \subsection{The Binomial Distribution}
                \begin{definition}
                    A random variable X is said to have a binomial distribution with parameters n and p 
                    if $P_X(x) = P[X = x] = \binom{n}{x}p^x(1-p)^{n-x}$ for x = 0,1,...,n and $0 < p < 1$
                \end{definition}

                \begin{note}\tab
                    \begin{enumerate}
                        \item Clearly $P_X(x) \geq 0$ but $\sum\limits_{\text{all x}} P_X(x) \stackrel{?}{=} 1$ 
                                \begin{proof}
                                    \begin{align*}
                                        \intertext{consider} 
                                        \sum_{x=0}^n \binom{n}{x}p^x(1-p)^{n-x} &= [p + (1-p)]^n \tag{by the binomial theorem}\\
                                    \end{align*}
                                    (to be continued...)
                                \end{proof}
                    \end{enumerate}
                \end{note}
                \newpage

    \section{Lec 12, Feb 14}
        \textbf{Midterm include materials up untill and include this class}
        How does the binomial distribution most often arrive? 
        \subsection{Binomial Setup}
        \begin{enumerate}
            \item We assume that we have n independent Bernoulli trials\\
                    (A Bernouilli trial is the one that can result in exactly one of two possible outcome.
                    By convention, we call these outcomes "success" and "Failure")
            \item We assume that if $S_i$ is the event success at trial i and $F_i$ as failure, \\
                    then P[$S_i$] = p for all i = 1, ..., n and P[$F_i$] =  1- P[$S_i$]  = 1-p.\\
                    Note that p is assumed to be constant from trial to trial \\
                    Let X = the number of observed successes in these n trials 
                    \begin{theorem}
                        $P[ X = x] = \binom{n}{x} p^x (1-p)^{n-x}$ for x = 0, 1 , ..., n
                    \end{theorem}
                    \begin{proof}\tab 
                        \begin{itemize}
                            \item Done already in the constant of the gene mutation problem when sampling from a very population (small sample) 
                            \item \textbf{Idea:} Find the probability of a specific configuration with x successes and n-x failures. 
                                    This is $$p^x(1-p)^{n-x}$$
                                    Every such configuration was this probability P[X=x] = sum of the probabilities of all such configurations\\
                                    THere are $\binom{n}{x}$ terms in this sum 
                            \item Hence in a word problem if 
                                    \begin{enumerate}
                                        \item you have trials that result in one of two possible outcomes with equal probability
                                        \item you can assume that these trials are independent, then X = number of successes in these n-trials has a binomial distribution
                                    \end{enumerate}
                        \end{itemize}
                    \end{proof}

            \item Be careful of immediately invoking the binomial distribution because you see 2 types of outcomes on each trial. The trials may be dependent. 
                    \begin{example}
                        Suppose that patients undergoing a certain treatment can survive $>$ 5 years or $\leq$ 5 years. \\
                        Suppose proportion of patients that survive more than 5 years is 0.80 \\
                        If 30 patients are to recieve this treatment. \\
                        What is the probability that at least 3 will survive more than 5 years. 
                    \end{example}
                    \begin{solution} \tab 
                        \begin{itemize}
                            \item Let X = number of patients who survive more than 5 years\\
                                    Assume that patients survive independent of each other \\
                                    Let $S_i$:= event that patient i survives more than 5 years i = 1, 2,..., 30 \\
                                    We can assume the binomial setup with n = 30 and P[$S_i$] = 0.80
                            \item Hence 
                                    \begin{align*}
                                        P[X \geq 3] &= \sum_{x = 3}^{30} P[X = x] \\
                                                    &= \sum_{x = 3}^{30} \binom{30}{x}(0.8)^x(1-0.8)^{30-x}
                                        \intertext{OR}
                                        P[X \geq 3] &= 1 - P[X < 3] \\
                                                    &= 1 - P[X \leq 2] \\
                                                    &= 1 - \sum_{x = 0}^{2} \binom{30}{x}(0.8)^x(0.2)^{30 - x}
                                    \end{align*}
                        \end{itemize}
                        \tab \tab \tab \tab \tab \tab \tab \tab \tab \tab \tab \tab \tab \tab \tab \tab \textbf{Q.E.D}
                    \end{solution}
        \end{enumerate}

        \subsection{\textbf{The geometric distribution}}
            \begin{definition}
                A random variables X is said to have geometric distribution with parameter $0 < p < 1$ if $P_X(x) = P[X = x] = (1-p)^{x-1}p$ x = 1, 2, ...
            \end{definition}
            How does the geometric distribution arrive ? \\
            \textbf{Answer:} 
                \begin{itemize}
                    \item Imagine a sequence of independent Bernouilli trial, with P[$S_i$] = p for i=1,2...
                    \item Let x = the trial at which the first success is observed 
                \end{itemize}
            \textbf{Claim:} $P[X=x] = (1-p)^{x-1}p$ for x =1,2,...\\
            \tab \tab (i.e the geometric distribution is the distribution of the "time" to success in a sequence of\\
            \tab \tab \tab  independent Bernouilli Trials with constant probability of success)
            \begin{proof}\tab 
                \begin{itemize}
                    \item The event \{X = x\} is equivalent to the event \{$F_1 \cap F_2 \cap ... \cap F_{x-1} \cap S_x$\}
                            \begin{align*}
                                P[X=x]  &= P[F_1 \cap F_2 \cap ... \cap F_{x-1} \cap S_x] \\
                                        &= P[F_1]*P[F_2]*...*P[F_{x-1}]*P[S_x] \tag{By the assumption that the trials are indep.}\\
                                        &= (1-p)^{x-1}p \tag{for all x = 1, 2, ...}
                            \end{align*}
                \end{itemize}
            \end{proof} 
            \begin{note}\tab
                \begin{itemize}
                    \item The Binomial gives the number of successes in a fixed number of trials n. 
                    \item The geometric gives the trial at which th first success occurs. 
                    \item The number of trials is not fixed 
                \end{itemize}
            \end{note}
        
        \subsection{Negative Binomial Distribution}
            \begin{definition}\tab
                \begin{itemize}
                    \item A random variable X has a negative binomial distribution with parameter K and p
                            if X describes the trial at which the kth success occurs in a sequence 
                            by independent Bernouilli trials with constant probability of success p.
                \end{itemize}
            \end{definition}

            Let's find $P_X(x) = P[X = x]$ for x = k, k+1, ...
            \begin{align*}
                P[X=x]  &= P[\text{ getting k-1 successes in x-1 trials $\cap$ getting k success on trial x}] \\
                        &= P[A \cap B] \\
                        &= P[B \mid A]* P(A) \\
                        &= P(B)*P(A) \tag{by independence} \\
                        &= \underbrace{p}_{P(B)}  \binom{x-1}{k-1} p^{k-1} (1-p)^{x-1-(k-1)} \tag{for x = k, k+1,...}\\
                        &= \binom{x-1}{k-1}p^k(1-p)^{x-k} \tag{for x = k, k+1, ...}
            \end{align*}
            \newpage
            
    \section{Lec 13, Feb 19}
        \subsection{Poisson Distribution}
            \begin{definition}
                The random variable X is said to have a Poisson distribution with parameter $\lambda > 0$ if and only if $$P_X(x) = P[X=x] = \frac{\lambda^x e^{-\lambda}}{x!} \tab \forall x = 0,1,...$$
                Notice
                \begin{align*}
                    \sum_{x=0}^{\infty} P_X(x)  &= \sum_{x=0}^{\infty}  \frac{\lambda^x e^{-\lambda}}{x!} \\
                                                &= e^{-\lambda} \underbrace{\sum_{x=0}^{\infty} \frac{\lambda^x}{x!}}_{e^{\lambda}} \\
                                                &= e^{-\lambda}e^{\lambda} \\
                                                &= 1
                \end{align*}
            \end{definition}
            \begin{note}
                The Poisson Distribution arises as an approximation to the Binomial distribution if "n is large and p is small" 
            \end{note}

            \begin{theorem}
                \textbf{(Poisson approximation to the Binomial)} \\
                Let X have a Binomial distribution with parameters n and p. Then 
                $$P[X=x]\rightarrow \frac{\lambda^x e^{-\lambda}}{x!} \tab \textbf{ as } n \rightarrow \infty, p \rightarrow 0 \tab \forall x = 0,1...$$
                such that np = $\lambda$ is constant

                \begin{proof}
                    \begin{align*}
                        P[X= x]     &= \binom{n}{x} p^x (1-p)^{n-x} \tag{for x = 0,1,...} \\
                                    &= \frac{n!}{x!(n-x)!}* \frac{p^x (1-p)^n}{(1-p)^x} \\
                                    &= \frac{n*(n-1)*...*2*1}{x!(n-x)*(n-x-1)*...*2*1}  * \big(\frac{\lambda}{n}\big)^x * \big(1 - \frac{\lambda}{n}\big)^n *\frac{1}{(1 - \frac{\lambda}{n})^x}    \tag{note $\lambda = np \Rightarrow p = \lambda/n$}\\
                                    &= \frac{\lambda^x}{x!} * \frac{\overbrace{n*(n-1)*...*2*1}^{\text{x terms}}}{\underbrace{n* n* ....* n}_{\text{x terms} (n^x)}} * \big(1 - \frac{\lambda}{n}\big)^n * \frac{1}{(1 - \frac{\lambda}{n})^x}  \\
                        \intertext{Now let $n \rightarrow \infty $}
                        \intertext{The first term on the R.H.S $\rightarrow$ 1 as $n \rightarrow \infty$ and each "ratio" $\rightarrow$ 1 (e.g $\frac{n-1}{n} \rightarrow 1$ as $n \rightarrow \infty$)}
                        \intertext{The second term on the R.H.S $\rightarrow 1$ as $n \rightarrow \infty$ since $(1 - \lambda/n \rightarrow 1)$ as $n \rightarrow \infty$ } 
                        \intertext{Finally $(1 - \lambda/n)^n \rightarrow e^{-\lambda}$ as $n \rightarrow \infty$ (another definition of $e^{-\lambda}$)}
                        \intertext{Hence} 
                        P[X= x]     &\rightarrow \frac{\lambda^x e^{-\lambda}}{x!} \tag{for x = 0,1,...}
                    \end{align*}
                \end{proof}
            \end{theorem}

            \begin{example}
                Suppose that sections of textile of length 1cm have a flaw in them with probability .01. 
                If 1000 such sections are examined. What is the approximate probability that at least 50 will have a flaw?
            \end{example}
            \begin{solution} \tab
                \begin{itemize}
                    \item Let X be the number of 1cm length sections that have a flaw
                    \item Assume the binomial setup so that X has binomial distribution with n = 1000 and p = 0.01
                    \item Hence since n is "large" and p is "small", we can use the Poisson approximation to the Binomial distribution
                    \item Exact answer(Binomial)
                        \begin{align*}
                            P[X \geq 50]    &= \sum_{x=50}^{1000} \binom{1000}{x} (0.01)^x(1-0.01)^{1000-x}
                        \end{align*}
                    \item Poisson Approximation: 
                        \begin{itemize}
                            \item Let $\lambda = np = 1000*0.01 = 10$ Then 
                            \begin{align*}
                                P[X \geq 50]    &= \sum_{x=50}^{\infty} \frac{10^xe^{-10}}{x!} \\
                            \end{align*}
                        \end{itemize}
                \end{itemize}
            \end{solution}

        \subsection{The Hypergeometric Distribution} 
            \begin{definition}
                A random variable X is said to hvae a Hypergeometric distribution with parameter N, a and n
                if and only if \[P[X = x] = \frac{\binom{a}{x}\binom{N-a}{n-x}}{\binom{N}{n}} \tab \forall x = 0, 1,...\]
            \end{definition}
            \begin{note}\tab
                \begin{itemize}
                    \item We see immediately that this is just the distribution of the number of tagged fish in a sample of size n 
                            drawn without replacement from a lake with N fish of which a are tagged (Fish in the lake problem) 
                    \item In fact, the Hypergeometric distribution describes the number of Type 1 objects observed in a sample of size n drawn without replacement from a "box" with N objects 
                            (a of Type 1 and N-a of Type 2) (Justitication was done earlier)
                \end{itemize}
            \end{note}
            \tab \\
            \textbf{\framebox{Notation:}} We have notation for describing the distribution of a named random variable 
                \begin{itemize}
                    \item X $\sim$ Ber(p): X has bernoulli distribution with paramenter p 
                    \item X $\sim$ Bin(n, p): 
                    \item X $\sim$ Poisson( $\lambda$ )
                    \item X $\sim$ Geometric(p)
                    \item X $\sim$ NegBin(n, p, k)
                \end{itemize}
           

        \subsection{Mathematical Expectation \& Variance}
            \begin{itemize}
                \item[\textbf{\framebox{Idea:}}] The probability distribution of a discrete random variable X, tells the whole story about X 
                                                    i.e How its value are distributed, with different probability. \\
                                                    However, maybe you want one of two summaries of $P_X(x)$, that captures the main features of this distribution.\\
                                                    The two features that immediately comes to mind are: 
                                                    \begin{enumerate}
                                                        \item The "centre" of the distribution or "average value" of X
                                                        \item The "variability" or "spread" of the values of X
                                                    \end{enumerate}
            \end{itemize}
            \begin{definition}(Expectation)
                Let X be a discrete random variable, Let E(X) = $\sum\limits_{\text{all x}} x P[X=x]$. 
                Then call E(X) the expected value of X (or the expectation of X)  
            \end{definition}
            \begin{note}\tab 
                \begin{enumerate}
                    \item Often we denote E(X) by $\mu_X$
                    \item we also call E(X) the population mean of X
                    \item E(X) can be thought of as the weighted average of X, with weights P[X=x]
                \end{enumerate}
            \end{note}

\end{document}
